\documentclass[12pt]{article}
\usepackage{amsmath}
\newcommand{\myvec}[1]{\ensuremath{\begin{pmatrix}#1\end{pmatrix}}}
\newcommand{\mydet}[1]{\ensuremath{\begin{vmatrix}#1\end{vmatrix}}}
\newcommand{\solution}{\noindent \textbf{Solution: }}
\providecommand{\brak}[1]{\ensuremath{\left(#1\right)}}
\providecommand{\norm}[1]{\lVert#1\right\rVert}
\let\vec\mathbf


\title{Coordinate Geometry}
\author{Charan(charan.n@sriprakashschools.com)}

\begin{document}
\maketitle
\section*{Class 10$^{th}$ Maths - Chapter 7}
This is Problem-5 from Exercise 7.3
\item  QUESTION: Find the area of the quadrilateral whose taken in order are A(-4,-2), B(-3,-5), C(3,-2) and D(2,3).




\solution \\
\\We have two triangles ABC and ADC.
Then,
\begin{align}
\\First consider triangle ABC 

\end{align}
\begin{align}
\\Area of triangle ABC&=\frac{1}{2}\mydet{ \myvec {AB\times BC}}\\&=\frac{1}{2}\mydet{ {-1}&{-6}\\{3}&{-3}}\\ &=\frac{1}{2}\left((3)+(18)\right)\\&=\frac{1}{2}(21)\\&=21/2 sq units
\end{align}
\begin{align}
\\Now, area of triangle ADC
\end{align}
\begin{align}
\\Area of triangle ACD&=\frac{1}{2}\mydet{ \myvec {AD\times DC}}\\&=\frac{1}{2}\mydet{ {-6}&{-1} \\{5}&{-5}}\\ &=\frac{1}{2}\left((30)+(5)\right)\\&=\frac{1}{2}(35)\\&=35/2 square units
\\NOW,AREA of quadrilateral = area of ABC + area of ADC
\\AREA = 21/2 + 35/2\\
\\AREA = 28 sq.units
\end{align}


\end{document}


